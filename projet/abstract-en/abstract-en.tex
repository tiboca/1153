\input{../head.tex}

\begin{abstract-en}
The loud-speaker is a tool that transforms an electric signal into a sound.  Thanks to a fixed coil, a mobile coil can oscillate and because it is attached to a membrane, it creates a sound wave.  
Many tests have been done to get a functional loud-speaker.  We did mathematical calculations but we had to use hypothesis that are simpler than the reality.  So the only way to know if it works or not is to test it.
In this document, calculations and the ideas to could have done such a tool are present.  
Our loud-speaker is able to produce a sound with a frequency between \SI{500}{\hertz} and \SI{5000}{\hertz} and has a power of \SI{2.5}{\watt}.  We made the loud-speaker with a membrane of \SI{0.17}{\meter} of diameter, a fixed coil of \numprint{450} spires and mobile one of \numprint{100} spires.  With the jack plug ( a cable that can rely a Gsm of a smartphone) we can get a sound and moreover its intensity can be modified.


In this course, \textit{Projet 2}, it has been asked to make a loud-speaker that can be connected to a smartphone with a Jack-plug.

To achieve this task, we had to work out to find a way to go through it.  We had to find simpler mathematical and physica models.  
The real problem was to tricky for us so we had to make some simpler assumptions.  However, we couldn't make random assumption because the theorie has to fit with the test we did in the lab.

In this document, calculations and the ideas that are necessary to make such a tool, are present.

Even if our loud-speaker doesn't work as well as we wanted, we learned so much from this challenge.

Our state of mind is pretty hard to describe: we are disappointed with the actual loud-speaker but for the rest the group is happy enough.


\end{abstract-en}

\input{../foot.tex}
